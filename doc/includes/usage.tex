Each resource (without NIC) simulates localy regardless to the others resources.
Only the NIC are "connected" to the others resources.

Without notion of global simulation time but only local simulation time, 
the different nodes simulates locally what they have to simulate, they send a message 
when they need it and receive a message when a message comes without synchronization
with the others nodes. So a basic NIC can be:
\begin{verbatim}
while not(end_of_simulation) {
    if there_is_a_message_to_send() {
        xsim_msg_send(&message);
    }
    if there_is_a_message_to_receive() {
        xsim_msg_recv(&message);
        handle_msg(&message);
    }
    update_its_time_in_order_to_stay_synchrone_with_its_resource();
}
\end{verbatim}


As we will see later, when the time management is added, a part of the NIC
simulation is made in Xsim. Moreover for each NIC, there is a FIFO which represents
all the messages it has to send. The simulation loop is:
\begin{verbatim}
while not(end_of_simulation) {
    simulate_the_components();
    forall NIC:i {
        simulate_the_NIC_send(to_send[i]);
        simulate_the_NIC_recv();
    }
}
\end{verbatim}


The user has to implement the simulation of the components of the resource.
For the NIC, the communication with the other components of the resource and 
the internal functioning must be also implemented. The communication with the
other nodes is done by Xsim.
In other words, once a message is put in \verb|to_send| list, it is sent as 
soon as possible and Xsim handles this
message until it arrives to its destination at the right simulation time.



