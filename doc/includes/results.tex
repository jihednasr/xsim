In order to evaluate the behavior of the system, some measures are taken.
They are divided in two categories: the global measures and the specialized 
measures. 
\begin{itemize}
    \item The global measure is useful to characterized the time performance of 
    the simulation or the listener. Time function provided by \verb|<time.h>| as
    \verb|clock_gettime| are used.
    \item The specialized measures measure some short part of the code as the 
    lock. The instruction \verb|RDTSC| and \verb|RDTSCP| are used in order to
    have bus cycle accuracy.
\end{itemize}

For the experimences, we used ... (décrire l'ordinateur hôte et le test que l'on fait tourner dessus).

We focus now on the global behavior of the system. First, the graphic~\ref{global_proc}
shows the evolution of the execution time when the number of processors running goes from 1 to X. 
We expected ...

We observe ...


\begin{figure}[h]
\begin{center}
    \includegraphics[width=0.7\columnwidth]{../gnuplot/pdf_gnuplot/gnuplot_test_global_processus_time.pdf}
	\caption{Processes execution time}
	\label{global_proc}
\end{center}
\end{figure}


The speed-up of the system is showned in graphic~\ref{speed_up_global}.
We expected ...

We observe ...


\begin{figure}[h]
\begin{center}
    \includegraphics[width=0.7\columnwidth]{../gnuplot/pdf_gnuplot/gnuplot_test_speedup_global_processus_time.pdf}
	\caption{Speed-up}
	\label{speed_up_global}
\end{center}
\end{figure}

This can be explain by concurrency limit because some locks are used ... graph~\ref{lock_wait}.


\begin{figure}[h]
\begin{center}
    \includegraphics[width=0.7\columnwidth]{../gnuplot/pdf_gnuplot/gnuplot_test_lock_wait.pdf}
	\caption{Evolution of the time spend waiting a lock}
	\label{lock_wait}
\end{center}
\end{figure}

