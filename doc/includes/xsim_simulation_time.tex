Each node can simulate faster or slower depending of what they have to
simulate. This can create causality error when a node which goes faster
receive a message in its past. In Xsim, the simulation time management
is added in order to avoid the causality error. For that, each node
must have a knowledge of the simulation time of the others nodes and
check if it can go further in its simulation without risk.

A pessimist method based on the blocking table and null messages to 
avoid deadlock was chosen.


%%%%%%%%%%%%%%%%%%%%%%%%%%%%%%%%%%%%%%%%%%%%%%%%%%%%%%%%%%%%%%%%%%%%%%%%%%%%%%%

\section{How it works}

We will adopt the point of view of a node. The node has to know how far the
other nodes are in their simulation and if they send messages to it.
But there may be multiple interfaces and on each interfaces, they could be
communication. So we need to maintain a table per interface of a node.
Each table contains what the interface n$^{\circ}i$ of a node knows
about the next time the interfaces n$^{\circ}i$ of the others nodes 
can send a new message. There is 2 possibilities for this time:
\begin{itemize}
    \item a node is sending a message on the interface n$^{\circ}i$ up to
    the simulation time $t_{end\_send}$, so it can not send a new message before
    $t_{end\_send}$.
    \item a node is not sending a message on interface n$^{\circ}i$, so 
    the next new message can be sent immediately. The next time the node can 
    send a message on this interface is its current simulation time.
\end{itemize}


We have to answer to 3 questions: 
how the message are managed with the simulation time,
when can a node advance its simulation time without risk and
how can we maintains the time tables?


\subsection{How are the messages managed with the simulation time?}

We will first define the send time and the arrival time.
The send time of a message corresponds to the simulation time when the 
first flit of the message is sent in the network. 
The arrival time correspond to the simulation time when the first flit
of the message is received by the addressee NIC.\\

A NIC can not receive
several messages in the same time. So if the first flit of a message arrives 
at simulation time $t$ and this message is $n$ flits long, the NIC knows that 
it can not receive an other message before $t+n$. 
It is the same in other sens: when the NIC send a message of size $n$ flits at time $t$,
it knows that it can not send an other message before $t+n$.
But a NIC can send a message and receive an other message in the same time. So 
we have to take care of both (reception and emission) independently.

A goal of Xsim is to simulate correctly the message passing in order to avoid
message overlapping. The policy for that is that messages are received in the order
of their \emph{due time}. The \emph{due time} is the arrival time of the message
if there is no other communication in the network during the time the message is
exchanged. So \verb|due_time = send_time + travel_time_from_the_sender_to_the_addresse|.
If 2 messages are overlapping, the arrival time of the message with the greater 
due time is shifted latter. 

There is no guaranty about the reception order of messages with the same 
due time and the same receiver on the same interface.\\

Moreover, as it is said before, a NIC of a node which send a message of size $n$ at 
time $t$ can not send an other message on its interface before $n+t$.
And a NIC of a node which receives a message of size $n$ at time
$t$ can not receive an other message from its interface before time $n+t$.


\subsection{When can a node advance its simulation time without risk?}
The node can advance its time when it knows that:
\begin{itemize}
	\item it can receive a message only in its future on all its NICs
    \item and it has nothing more to do in its current simulation time.
\end{itemize}
In this case, the node can jump to the minimum time between:
\begin{itemize}
    \item next time it can (possibly or certainly) receive a message from the network
    \item next time it has a message to send 
    \item next time it has a component to simulate.
\end{itemize}

We will now explain the two first cases:
\begin{itemize}
    \item Xsim is built in such a way that the messages go as fast as possible from the 
    send queue of the sender interface to the receive queue of the receiver interface.
    In the receive queue, they are stored until the node arrives to the right simulation
    time to really receive the messages. So it is possible that a message which should
    be delivered at time $t_{delivered}$ is in the receive queue at time $t_{sim}$ with
    $t_{sim} < t_{delivered}$.

    So the minimum time of all the time tables of a node (+ the travel time) gives the
    next \emph{possible} message receive time. But if there is already a message in 
    one of the receive queue of the node, the message has the smallest arrival time
    among all the messages already in the receive queues of the node and the arrival 
    time of this message is fewer than the \emph{possible} next message receive time, 
    in this case we have the next \emph{certain} message receive time.

    The third possibility is that an interface can not receive a message before $t$.
    So in all the case, the next receive time must be greater or equal to $t$.

    The formula used is:
\begin{verbatim}
for all the interface i, take the minimum of:
    MAX(
       -Next time interface i can receive a message,
       -MIN(
           -next possible message receive time on interface i,
           -next certain  message receive time on interface i if it exists
           )
       )
\end{verbatim}

    \item If a NIC has messages to send, it must wait the last message is completely
    send before it can send a new message. If it has no message to send, this value
    is ignored or $+\infty$.

\end{itemize}


There is no risk because we jump to the next time it is possible that it arrives 
something or we do something. 
Nevertheless there is one case which could seem not safe: 
the next time the node could receive a message is $t_{pr}$,
the next time an interface $i$ is certain to deliver a message $m$ is $t_r$,
$t_r>t_{pr}$ and we jump to $t_r$ because all the other interfaces of
the node can not receive a message before a time greater than $t_r$.
There is a simulation time between $t_{pr}$ and $t_r$ which seems risky
as shown in figure~\ref{risky}.

\begin{figure}[h]
\begin{center}
    \includegraphics[width=0.4\columnwidth]{schema/Risky.pdf}
	\caption{Risky time interval}
	\label{risky}
\end{center}
\end{figure}

We can receive a message $m'$ with a due time smaller than the due time of $m$
between $t_{pr}$ and $t_r$. $m'$ must be delivered before $m$ with regard to
the deliver order given earlier.
So we avoid the interface to deliver a message if the next possible message 
receive time is smaller than the current simulation time.


\subsection{How we maintain the time table?}
First we need to know the time a message takes from one point to an other point
in the network. We will call it \verb|travel_time(i, j)| to indicate the time
a message takes from node $i$ to node $j$.\\

We can update the time table of an interface when a message from an interface 
(with network identifier $id$ and belonging to the node $n$) send a message. 
Indeed, when a message is posted, all the listeners read it and so this message
informs the nodes about where is the sender node in its simulation. \\

There are 2 different kind of update.
The first update concerns the listener with the same network number.
The $listener$ of our node $k$ reads this message and indicates to its interface 
corresponding to $id$ to update its time table. For that this computation is made:
\begin{verbatim}
time_table_of_id[n] = msg->send_time + (msg->nb_flit);
\end{verbatim}
This computation reflects the fact that the node $n$ can not send an other 
message on the interface number $id$ before it terminates the sending of 
the current message.

Normally the message should be read in time increasing order compare to the sender
node (cf \ref{internal_list} for more detail about the message order). 
This guaranties the fact that the value of the time table can only increase.\\

The second update concerns the time tables of the other interfaces of node $k$. The computation is
in this case:
\begin{verbatim}
time_table[n] = MAX(time_table[n], msg->send_time);
\end{verbatim}
Indeed it could be that the node $n$ has update its time since the last time it
sent a message on the other interface. \verb|msg->send_time| correspond to the time
the node $n$ sent this message but also to the simulation time of node $n$ when it
send this message. 
A node can not send a message in its past, so we are sure that the next possible
message send on any interface is at least at time \verb|msg->send_time|.
And we take the \verb|max| because it could be that the interface has more
accurate information coming from an other message.


%%%%%%%%%%%%%%%%%%%%%%%%%%%%%%%%%%%%%%%%%%%%%%%%%%%%%%%%%%%%%%%%%%%%%%%%%%%%%%%

\section{Politic for the time propagation information}

Two politics for the propagation of the simulation time information are implemented.
The first is a systematic propagation of the time information, 
the second is a propagation based on demand-driven time information.

For that a field \verb|type| is added to the message in order to indicate what is
the purpose of the message. The message can be:
\begin{itemize}
    \item a null message: the goal of this message is only to propagate the time information
    to all the nodes.
    \item a time request: Ask the other nodes to send a null message because
    the sender node is blocked and needs their time information to progress.
    \item end of simulation: inform the other nodes that the sender node is not more
    connected to this network and so will never send a message in the future in this
    network. This append when a node is not connected to a network or when it finishes
    its simulation and so is disconnected from all networks.
    \item a message with payload: it is a normal message with payload and a 
    not null.
\end{itemize}

\subsection{Systematic propagation}
It is the simplest way to communicate its time simulation progression.
Indeed after each time jump, the node send a null message in order all the other
node knows its current simulation time.

This method is also the quicker way to inform the other node about our progress.

\subsection{demand-driven propagation}
When a node can not make a time progress because it has not enough information
about the other nodes, it send a time request. When
a node receives a time request, it answers with a null message.\\

Some optimisations are done in order to avoid to send a lot of unnecessary 
null message and time request.

When a node send a time request, it informs also the other nodes about its
current simulation time like a null message.
A node which has already send a message (null message, request, normal, ...) 
at simulation time $t$ does not send a null message if it receives a time 
request and is still at time $t$.
A node send only one time request for a given simulation time.\\

This politic can slow down the simulation but can not created deadlock.
Indeed if there is some simulation time progress of one node that we do not know,
we will know it with a time request or because it send a message.\\

A tricky case can occur when: we are node $n$ at simulation time $t$ and
we send a time request because we can not make a time progress because the other
nodes are late compare to us. But all the other nodes have already send a message
at their current simulation time so they do not answer to the time request
(and even if they do, it does not give information to node $n$ that it has not already).
So node $n$ can not simulate further for the moment.

The other nodes makes progress but does not send messages. So node $n$ is still blocked.
But at some point, the other nodes will be blocked because node $n$ is not enough far in
the simulation and they send a time request. At this point node $n$ knows the time the
other nodes and that it can not receive a message before $t'$ with $t'>t$. So it can 
continue its simulation.
