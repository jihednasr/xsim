A Chip Multi-Processors (CMP) is composed of multiple resources and networks.
A resource can be a processor core, a DSP, an FPGA block, a memory bock, 
an other Intellectual Properties (IP), or a mixed of all that. Each resource 
has the size of a synchronous domain.
The resources are connected to a network via a \emph{Network interface 
component} (NIC). A NIC can be connected to only one network but a CMP can have 
multiple networks. So each resource has as
many NIC as there are networks to be connected with. Figure~\ref{CMP_schema} 
represents a basic CMP with two 2D mesh network.

\begin{figure}[h]
\begin{center}
    \includegraphics[width=0.6\columnwidth]{schema/CMP_schema.pdf}
	\caption{Chip Multi-Processors scheme}
	\label{CMP_schema}
\end{center}
\end{figure}


Xsim simulates the networks of a CMP. It abstracts the networks, the routers, 
the FIFOs and gives an interface to the NIC in order they can send and receive
messages. Figure~\ref{Xsim_conceptual_view} illustrates this description.

\begin{figure}[h]
\begin{center}
    \includegraphics[width=\columnwidth]{schema/Xsim_conceptual_view.pdf}
	\caption{Conceptual view of Xsim}
	\label{Xsim_conceptual_view}
\end{center}
\end{figure}


In Xsim, an interface is used to represent the link between a NIC of a resource
and a network. The network is internally represented by an interface identity
number. The NIC can send and receive messages with corresponding function invocation.
A \emph{node} represents one resource and its links to all the networks.

%%%%%%%%%%%%%%%%%%%%%%%%%%%%%%%%%%%%%%%%%%%%%%%%%%%%%%%%%%%%%%%%%%%%%%%%%%%%%%%%%%%

\section{Internal structures}
Xsim should simulate the networks. The latency in the communication between 
the nodes is a part of it. Indeed
a message can be sent from one node to another almost instantaneously for the 
software (time of a \verb|memcpy| or time to give a pointer to the message), 
but in reality there is delay for 
communication between nodes. Xsim must simulate this delay. It is why
Xsim is composed of multiple lists to store the messages before it can deliver
the messages to the right NIC. 

All the interfaces of
a node shared the same list to store the messages to be sent but they have their
own list for the received messages. Moreover there is a central list which 
collects all the messages. The different nodes read it regularly in order to 
see if there is messages for one of their interfaces. This list plays a little 
the role of a central router.
These internal structures are shown in figure~\ref{Xsim_internal_structures}.

The central list is used to broadcast the information to all the node with no
more cost as a simple sent and to simplify some race problems which can occur
if the messages were directly posted by the sender in the reception list of
the receiver. This broadcast property is useful for the time management which
is presented later.

\begin{figure}[h]
\begin{center}
    \includegraphics[width=\columnwidth]{schema/Xsim_internal_structures.pdf}
	\caption{Internal structures of Xsim}
	\label{Xsim_internal_structures}
\end{center}
\end{figure}

%%%%%%%%%%%%%%%%%%%%%%%%%%%%%%%%%%%%%%%%%%%%%%%%%%%%%%%%%%%%%%%%%%%%%%%%%%%%%%%%%%%%

\section{Management of a message}

We used what we call a \emph{listener} to manage the messages inside Xsim.
There is one listener by node.

A \verb|xsim_msg_send()| call done by a NIC stores the message in the sent list of the 
corresponding node. After this message is copied by the listener of the node 
in the central list where all the listener come read it. The listener of the 
destination nodes copies it in the receive list of the appropriate interface. 
The NIC of this interface can then read this message when it is the good time
with \verb|xsim_msg_recv()|.



%%%%%%%%%%%%%%%%%%%%%%%%%%%%%%%%%%%%%%%%%%%%%%%%%%%%%%%%%%%%%%%%%%%%%%%%%%%%%%%

\section{Properties of the internal structures}

\label{internal_list}
If we want to be accurate and add the simulation time management, we need 
to guaranty this property for the internal structure:
\begin{myproperty}[Main property]
\label{based_property}
The call to \verb|xsim_msg_recv()| always return messages with increasing arrival time
(but not strictly increasing) when it returns a message and the interfaces never
miss a message.
\end{myproperty}

To guaranty the property~\ref{based_property}, we need to define a partial order
and introduce other properties.
\begin{mydef}
For a node $n$ and 2 messages sent by $n$, 
$m^n_1$ and $m^n_2$, $m^n_2$ is strictly 
greater than $m^n_1$ if and only if $m^n_2$ was sent by the node $n$ 
after $m^n_1$.
\end{mydef}
We will write: $m^n_1\prec m^n_2$.
It is possible to have several messages sent in the same time by the same node if it
send them on different networks/interfaces. This messages are equal and there arrival
order has no importance.

\begin{myproperty}
\label{property2}
During a complete simulation, the set of all the messages posted are called $M$. 
All the listeners read the messages of $E$ in the messages box in the increasing 
order for the partial order. 
\end{myproperty}

\begin{myproperty}
\label{property3}
The reception queue is sorted
in the increasing arrival time order and it is read in the increasing order.
\end{myproperty}

Property~\ref{property2} is necessary for property~\ref{based_property}.
\begin{proof}
We assume that this property does not hold, the current simulation time of the 
node $k$ is $t$ and the interface $i$ of node $k$ is waiting the 
node $n$ (interface $i$ has already received a message from all the others nodes 
with an arrival time greater than $t+a$ from example, $a>0$).

Because the property does not hold, it is possible that the listener of node $k$ 
reads a message $m^n_2$ before $m^n_1$ with $m^n_1\prec m^n_2$, both send on the
same interface.
The listener puts $m^n_2$ on interface $i$ of node $k$. 
We assume that the arrival time of $m^n_2$ is $t+b$ with $0<b<a$.

So the reception part of the interface $i$ of node $k$ can jump to $t+a$ because the interface
believes that it can not receive a message before $t+a$. But $m^n_1\prec m^n_2$, 
so the arrival time of $m^n_2$ is greater than $m^n_1$.
So $m^n_1$ is missed.
Moreover, if at this moment a 
call to \verb|xsim_msg_recv()| on interface $i$ of node $k$ is done, $m^n_2$ is
return.

So this property is necessary.
\end{proof}

Property~\ref{property2} and~\ref{property3} implies property~\ref{based_property}.
\begin{proof}
Because all the messages of $M$ are read in the increasing order for the partial order,
they arrive in the reception queue of the interfaces in the right order. And because
of property~\ref{property3}, the messages are delivered in the right order and no 
messages are missed.
\end{proof}

To ensure property~\ref{property2}, it is chosen to have a sorted list in the partial order for the
messages box. But in this case it is also necessary to ensure the next property:
\begin{myproperty}
The messages are put in the central list in the partial order.
\end{myproperty}
\begin{proof}
If the list is sorted in the partial order and the messages are not put in the
partial order, it is possible to receive $m^n_2$ before $m^n_1$ in the list
with $m^n_1\prec m^n_2$. So if a listener comes read the
messages box before $m^n_1$ is put in the list, it reads $m^n_2$ but not $m^n_1$.
Property~\ref{property2} is not guarantied.
\end{proof}

We chose to use a FIFO for the messages box and for the send list of each node.
\begin{proof}
It is guarantied that the message was put in the send list in the partial order by definition.
(The moment the message is put in the send list must correspond to the send time.)
Then, the messages was taken in this list in the partial order and put in the messages box in
the partial order because this list is sorted in the partial order (FIFO) - so sorted in the send time order
because this list contain messages from only one node. Finally, the messages box is
sorted in the partial order (because FIFO) and so the listener reads the messages in
the partial order. So because of property~\ref{property2} and property~\ref{property3}, property~\ref{based_property}
is guarantied.
\end{proof}



%%%%%%%%%%%%%%%%%%%%%%%%%%%%%%%%%%%%%%%%%%%%%%%%%%%%%%%%%%%%%%%%%%%%%%%%%%%%%%%%%%%%

\section{Messages box}

The messages box is a FIFO list with a head and a tail. The insertion is done in tail.
A cell can be deleted only if all the listeners have read it. The suppression is defined later
and multiple cells can be deleted in one shot.\\

The cell is composed of a data field, a mark field and a next field. 
A listener puts its mark in the mark field iff it has read this cell and is jump to the 
next cell.

\begin{figure}[h]
\begin{center}
    \includegraphics[width=0.6\columnwidth]{schema/List_type.pdf}
	\caption{Scheme type of the list}
	\label{List_type}
\end{center}
\end{figure}

\subsection{Concurrent reading of the list by the listener}
The listener access to the list with a cell pointer \verb|current| which indicates
the last cell read by the listener. It reads the list in the list order, one cell
after another. When a listener wants to read the list, it begins
to do some special work for the deletion, which will be explained later. For the moment
the only thing we need to know is that this special work move \verb|current| one cell
forward.
The listener reads the cell, jumps to the next cell and marks the last, reads 
the current cell, ...
If the next cell is the tail, the listener does not jump and does not mark the 
current cell and stops to run through the list. 

Without the deletion at the beginning and the marking, the listeners are only reading.
In the loop, there is only the marking problem, so an atomic marking guaranties no 
concurrency problem in the reading of the list. 
The reading of a list by a listener can be resumed by this pseudo-code:
\begin{verbatim}
listener_read_list(xsim_msg_list_elt_t *current, xsim_msg_list_t *list)
{
    if (current->next != list->tail) {
        special_delete_work(current, list);
        read_msg(current);
        while(current->next != tail) {
            tmp = current;
            current = current->next;
            atomic_mark(tmp->mark);
            read_msg(current);
        }
    }
}
\end{verbatim}
The list must always contains at least a cell in addition to the head and the tail 
because the \verb|current| pointer of the listeners needs to point to a valid cell 
(not the head or the tail).



\subsection{Insertion}
The insertion needs two or more CAS and is done in tail.
It is based on the MS FIFO.
\begin{verbatim}
insertion(xsim_msg_list_elt_t *new_msg, xsim_msg_list_t *list)
{
    new_msg->next = list->tail;
    while (1) {
        last = list->tail->next;
        if (CAS(last->next, tail, new_msg)) {
            break;
        } else {
            CAS(list->tail->next, last, last->next);
        }
    }
        
    CAS(list->tail->next, last, new_msg);
}
\end{verbatim}

The first CAS ensures that no new cell can be added in the list if the tail pointer is
not up-to-date and that only one cell at a time can be added. 
The second and third CAS up-dates the tail pointer. In this way, the update of the
tail pointer does not depend of only one process.
Every process can up-date the tail pointer and so succeed the insertion in the next loop.
So several insertions can be try concurrently without problem and the insertion is lock-free.\\

The CAS is an atomic assembly instruction and is defined as:
\begin{verbatim}
CAS(value, old, new)
{
    if (value == old) {
        value = new;
        return true;
    } else {
        return false;
    }
}
\end{verbatim}


There is no problem with the concurrent reading of the list by the listeners 
because the reading listeners do not care of the tail's next field. 
%If a reading listener is not at the end of the list, the insertion can not be a problem. 
%If the listener is on the last cell before the tail, there is two possibility:
%\begin{itemize}
%	\item the reading listener does the comparison of the \verb|while| before the CAS of the 
%    insertion, so it jump out of the loop. No problem.
%	\item if the cell is added before the listener does the comparison for the \verb|while|,
%	so it executes the loop again and stop the next comparison if no other cell is added.
%\end{itemize}


\begin{figure}[h]
\begin{center}
    \includegraphics[width=0.6\columnwidth]{schema/List_insertion.pdf}
	\caption{Insertion of two cell in the same time - one has succeeded the CAS, the other not}
	\label{List_insertion}
\end{center}
\end{figure}


\subsection{Deletion}
The deletion can be done in only two case. When the listener begins to read a list,
it checks if all the others listeners have already read all the previous cells and
jumped to the next cells. For that it checks if it is 
the last to put its mark in the mark field of its current cell. If it is the case,
it deletes all the previous cell (its current cell included) and jumps to the next
cell, else it jumps immediately to the next cell and the delete job will be done 
by an other listener which is currently on a previous cell (its current cell included).
This is the most current case.

The other case for the deletion is a security. There is some special case
where deletion as presented above never arrives. A deletion can be try in the removal 
function of the garbage list (cf \ref{garbage list} for more information).


When a cell is deleted, it is put in an other list which is called a garbage list.
This list is a pool of cells where we can pick one when a listener needs to add a 
new cell in the message box. The addition of cells
can be done in block: the function needs only a pointer to the first cell and a
pointer to the last cell. (For more detail, see ~\ref{garbage list}.)

\begin{verbatim}
special_delete_work(xsim_msg_list_elt_t *current, xsim_msg_list_t *list)
{
    tmp       = current;
    current   = current->next;
    try_again = true;
    do {
        old_mark = tmp->mark;
        if (old_mark == its_mark) {
            /* 
             * All the other listener are already on the next cells.
             * We can delete this cell and the previous cells.
             */
            do {
                begin = list->head->next;
            } while (CAS(list->head->next, begin, current));
            garbage_list_add(begin, tmp);
            try_again = false;
        } else {
            new_mark = mark(old_mark);
            try_again = !CAS(old_mark, tmp->mark, new_mark);
        }
    } while (try_again);
}
\end{verbatim}

A pre-condition to the call to this function is that the next cell of \verb|current|
is not the tail.
We will now proof that this algorithm for the deletion works (no concurrency problem).
\begin{proof}
We take a listener of a node $n$. We will call it $L$. $L$ is currently on the cell $C$.
All the listener $All$ can be divided in 3 sets: 
\begin{itemize}
    \item $Prev$ - the \verb|current| pointer of all these listeners pointed on a cell
    which is before $C$ in the list.
    \item $Same$ - the \verb|current| pointer of all these listeners pointed on $C$ 
    in the list.
    \item $Next$ - the \verb|current| pointer of all these listeners pointed on a cell
    which is after $C$ in the list.
\end{itemize}

There are 2 different cases:
\begin{itemize}
    \item If the mark of $C$ is equal to the mark of $L$, it means
    that it stays only its mark to remove. It is equivalent to say that the other listeners
    marked already $C$. So $Same = L$, $Prev = \emptyset$ and $Next = All\setminus L$. 
    It is not possible that a listener is working on a previous cell, so we can
    remove all the cells before $C$ ($C$ included). Moreover if an other listener 
    is removing cells in the same time because the garbage list needs cells 
    (cf \ref{garbage list}), they will not remove the same cells because of the 
    CAS on the head of the list and \verb|tmp| cell is not marked by $L$.
    
    \item If $C$'s mark is different of $L$'s mark, it means there is other 
    listeners which have not jump to the successor cell of $C$. 
    So $Same \neq \{L\}$ or $Prev \neq \emptyset$.
    In the both case, $L$ tries to mark $C$ with a CAS. 
    
    If it succeeds, no other
    listener has marked $C$ during the time $L$ was trying to do it. It means
    the situation is staying the same: $Same \neq \{L\}$ or $Prev \neq \emptyset$.
    So $L$ goes read the next cell and does not delete cells.

    If the CAS failed, it means that an other listener marked $C$ during the
    time $L$ was trying to do it. The situation has maybe changed. 
    $L$ must do again the comparison to see if $Next = All\setminus L$ or not.

\end{itemize}
\end{proof}

%%%%%%%%%%%%%%%%%%%%%%%%%%%%%%%%%%%%%%%%%%%%%%%%%%%%%%%%%%%%%%%%%%%%%%%%%%%%%%%%%%%%%%%%%%%%%
\subsection{Condition for deletion}
\label{deletion_not_always_possible}

First, we can sort the listener: $L_1 \prec L_2$ if the \verb|current| pointer of $L_1$
is pointing to a cell which is before the cell pointed by the \verb|current|
pointer of $L_2$. $L_1 = L_2$ if their \verb|current| pointer is pointing to the same cell.

We proof now that:
\begin{myproperty}
If there is a set of listener so that:\\
all the listeners of this set are equal, smaller to all the other listeners which are not in the set
and all not currently reading the message box.
Then all the cells before or equal to the cell pointed by this set will be deleted.
\end{myproperty}
\begin{proof}

The deletion can be done when a listener is beginning to read a list.
There is 2 cases:
\begin{itemize}
    \item The set is composed only of one listener. This listener is already smaller 
    than all the other and is not performing the reading.
    When it begins the reading, it will perform the deletion.

    \item The set has several listener.
    The listeners will begin to read one after the other. And at some point,
    it will remain only one listener on this cell because the listener can
    mark the cell only one at a time. This listener will perform the deletion.
\end{itemize}
\end{proof}

But it exists some special case where the conditions for the deletion are never met.
One example is given in figures~\ref{sp1},~\ref{sp2},~\ref{sp3},~\ref{sp4},~\ref{sp4}.

\begin{figure}[h]
\begin{center}
    \includegraphics[width=0.5\columnwidth]{schema/special_case1.pdf}
	\caption{Initial state: 3 listeners (1 is reading, the 2 other not)}
    \label{sp1}

    \includegraphics[width=0.6\columnwidth]{schema/special_case2.pdf}
	\caption{2 cells are added by a listener which are not reading}
    \label{sp2}

    \includegraphics[width=0.6\columnwidth]{schema/special_case3.pdf}
	\caption{Listeners 1 and 2 are trying to perform the deletion but can not because listener 3 is on the same cell}
    \label{sp3}

    \includegraphics[width=0.6\columnwidth]{schema/special_case4.pdf}
	\caption{Listeners 1 and 2 advance}
    \label{sp4}

    \includegraphics[width=0.6\columnwidth]{schema/special_case5.pdf}
	\caption{Listener 3 advances, and listeners 2 and 3 stops reading. Same situation as the initial state.}
    \label{sp5}

\end{center}
\end{figure}

These cases should be normally rather rare,
but if it appends perpetually, there is no deletion.
It is why a special test and a possible deletion is added in the garbage list removal in order to
avoid problem due to the lack of cells deletion.


\subsection{Garbage list}
\label{garbage list}

We give 2 primitives to manipulate the garbage list:~\verb|garbage_list_add()|
and~\verb|garbage_list_remove()| which allow to add cells and to remove a cell
respectively. The addition of new cells is done in tail and the removal is done
in head.

\begin{verbatim}
garbage_list_add(xsim_msg_list_elt_t *begin, xsim_msg_list_elt_t *end)
{
    end->next = garbage_list->tail;
    while (1) {
        last = garbage_list->tail->next;
        if (CAS(last->next, garbage_list->tail, begin)) {
            break;
        } else {
            CAS(garbage_list->tail->next, last, last->next);
        }
    }
    CAS(garbage_list->tail->next, last, begin);
}
\end{verbatim}

\begin{verbatim}
xsim_msg_list_elt_t garbage_list_remove()
{
    do {
        remove_cell = garbage_list->head->next;
        if (garbage_list->head->next->next == garbage_list->tail) {
            collect_cells();
            if (garbage_list->head->next->next == garbage_list->tail)
                return NULL;
        }
    } while (!CAS(garbage_list->head->next, remove_cell, remove_cell->next));
    remove_cell->next = NULL;
    return remove_cell;
}
\end{verbatim}



We assume that: 
\begin{myproperty}
\label{garbage_list}
The garbage list has always at least one element without the head 
and the tail. 
\end{myproperty}
This property guaranties there is no interaction betwen the addition and the removal.
This property is checked before each removal.\\

For the add, the reasons why it works are the same as for the insertion in the message box.

The removal of one cell uses one CAS to ensure the absence of race problem. If the first
cell of the garbage list is removed during an other listener tries also to have a new
cell, the second CAS will fail and the second listener should try again to remove the 
new first cell.

%If a listener is removing one cell and an other listener is adding cells, it could be a 
%problem only in one case: the garbage list contains only one valid cell. If there is more
%cells, there is no interaction between cells removal and addition. But in reality, there is
%no race problem because of property~\ref{garbage_list}.
%There are two case:
%\begin{itemize}
%    \item the removal begins before the addition of cells, so the removal fails because
%    the list has only one valid cell. No race problem.
%    \item the addition begins before the removal. The second instruction must be already
%    executed in order the removal can see more than one valid cell. The removal can 
%    run normally. The last cell added can not be removed even if there is a lot of removal
%    in the same time because of property~\ref{garbage_list}. So the cell 
%    addition can be finished without problem.
%\end{itemize}

The goal of \verb|collect_cells()| is to collect the cells of the messages box which are 
already read by all the listener but not already inserted in the garbage list because we are
in a situation as describe in \ref{deletion_not_always_possible} and all the cells are used.
This function can be called by several listener but only one can succeed.

\begin{verbatim}
collect_cells()
{
    begin = message_box->head->next; 
    end = message_box->head;
    while (all_the_listener_have_marked(end->next)) {
        end = end->next;
    }
    if (end == message_box->head) {
        /* no cells to collect */
        return;
    }
    if (CAS(message_box->head->next, &begin, end->next)) {
        /* collect some cells */
        garbage_list_add(begin, end);
        return;
    } 
    /* a listener has already collected some cells */
    return;
}

\end{verbatim}

The last cell of the message box can never be marked by all the listener, so
\verb|end| can never go to the tail. The CAS fails only if an other listener 
remove cells from the messages box.

Moreover we can mark the mark field of the tail of the messages box and the garbage list 
in order the listener are obliged to stop at the end of a list even if the cells
are garbaged during the time it is traversing the message box.


\subsection{ABA problem}
